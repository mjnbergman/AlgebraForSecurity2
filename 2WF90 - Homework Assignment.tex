\documentclass{article}
\usepackage{graphicx}
\usepackage{tikz}
\usepackage{amsmath}
\usepackage{amssymb}

\begin{document}
\author{Maiko Johannes Nicolaas Bergman, 1259946 \\ Dustin Wilhelmus Maria Bessems, 1228685 \\ Sander Leon Maria Cauberg, 1008909 \\ Guy Jozef Corien Puts, 1232041 \\ Rick Stolk, 1263722}
\title{}
\date{\today}
\maketitle{2WF90 - Algebra for Security, Chapter 2.4 Exercise 12}
\newpage
\begin{enumerate}
\setcounter{enumi}{11}
\item Consider the element $a=X+(X^{3}+X+1)\mathbb{Q}[X] \textrm{ in } \mathbb{Q}[X]/(X^{3}+X+1)$.
\begin{enumerate}
\item We show that $X^{3}+X+1$ is irreducible in $\mathbb{Q}[X]$ and conclude that $\mathbb{Q}[X]/(X^{3}+X+1)$:
\newline
Let $m, n \in \mathbb{Z}$ be relatively prime, and let $f(X)=aX^{3}+bX^{2}+cX+d \in \mathbb{Q}[X]$ be a polynomial with $a, b, c, d \in \mathbb{Z}$.
\newline
We prove that if $m/n$ is a root of $f(X)$, then $m|d$ and $n|a$:
\newline
We assume that $\frac{m}{n}$ is a root of $f(X)$.
$\frac{m}{n}$ is a root of $f(X)$, so $f(\frac{m}{n})=0$.
\newline
\{Proof\}
\newline
So, we have proved that if $m/n$ is a root of $f(X)$, then $m|d$ and $n|a$.
\newline
\{Proof\}
\newline
So, we have proved that $f(X)=aX^{3}+bX^{2}+cX+d \in \mathbb{Q}[X]$ is irreducible. This applies for any $a, b, c \textrm{, and} d \in \mathbb{Z}$. If we fill in the formula with $a=1, b=0, c=1, d=1$, we get the polynomial  $X^{3}+X+1$.
\newline
So, we have proved that the polynomial $X^{3}+X+1$ is irreducible in $\mathbb{Q}[X]$.
We show that $\mathbb{Q}[X]/(X^{3}+X+1)$ is a field:
We have previously proved that the polynomial $X^{3}+X+1$ is irreducible in $\mathbb{Q}[X]$.
A theorem provided during the lectures states that: $\mathbb{K}[X]/f$ is a field if and only if $f$ is irreducible. If $f$ is our irreducible polynomial $X^{3}+X+1$, and $\mathbb{K}=\mathbb{Q}$, we can say that, by the given theorem, $\mathbb{Q}[X]/(X^{3}+X+1)$ is a field.
\newline
QED
\end{enumerate}
\end{enumerate}
\end{document}